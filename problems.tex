\documentclass[a4paper]{article}

\usepackage{amssymb}
\usepackage{color}
\usepackage{amstext}
\usepackage{fancyhdr}
\usepackage{amsmath}
\usepackage{float}
\usepackage{amsthm}

\usepackage[hmargin=3.5cm,vmargin=3.5cm]{geometry}
\newcommand{\eql}{\mathrel{\rotatebox[origin=c]{180}{$\ge$}}}
\pagenumbering{arabic}

\pagestyle{fancyplain}

\fancyhf{}

\lhead[Nathan Van Maastricht]{Nathan Van Maastricht}
\rhead[Problems]{Problems}
\cfoot[\thepage]{\thepage}

\begin{document}
\section{Lonely Runner Conjecture}
Given a unit length circulr track and $k$ runners that are travelling at a pairwise distinct speed. Call a runner \emph{lonely} if they are at least $1/k$ units from every other runner at some time $t$. The Lonely Runner Conjecture states that every runner is lonely at some time.

\section{$p^\alpha q^\beta$}
Fix two primes, $p$ and $q$, and take $\alpha,\beta\in\mathbb{N}$. Given a number of the form $p^\alpha q^\beta$ find the next number of the same form in polynomial time.

A generalisation of this problem. Fix primes $p_i$, and take $\alpha_i\in\mathbb{N}$ where $i$ ranges from $1$ to $n$ in both cases. Given a number of the form $$\prod_{i=1}^{n}p_{i}^{\alpha_i}$$ find the next number of the same form in polynomial time.
\end{document}
