\documentclass[a4paper]{article}

\usepackage{amssymb}
\usepackage{color}
\usepackage{amstext}
\usepackage{fancyhdr}
\usepackage{amsmath}
\usepackage{float}
\usepackage{amsthm}

\usepackage[hmargin=3.5cm,vmargin=3.5cm]{geometry}
\newcommand{\eql}{\mathrel{\rotatebox[origin=c]{180}{$\ge$}}}
\pagenumbering{arabic}

\pagestyle{fancyplain}

\fancyhf{}

\lhead[Nathan Van Maastricht]{Nathan Van Maastricht}
\rhead[Problems]{Problems}
\cfoot[\thepage]{\thepage}

\begin{document}
\section{Lonely Runner Conjecture}
Given a unit length circulr track and $k$ runners that are travelling at a pairwise distinct speed. Call a runner \emph{lonely} if they are at least $1/k$ units from every other runner at some time $t$. The Lonely Runner Conjecture states that every runner is lonely at some time.
\end{document}
